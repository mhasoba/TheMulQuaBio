\chapter{High Performance Computing}
\label{chap:HPC}

These instructions also apply, with suitable modifications, for R 
scripts.

\section{Preparing scripts for running on the HPC}

The script you will run needs a sha-bang (telling it what shell to run, 
usually bash), you need to allocate resources to PBS (such as walltime, 
number of processors, and memory\footnote{Most of the cx1 nodes have 
multiple cores, so there's no fixed memory assigned to each core. If 
you use more memory than your request on your \#PBS directive, your job 
is likely to be terminated. If you request more memory than is 
available, the job will remain queued until sufficient memory is free 
for the job to run}, using the {\tt \#PBS} directive), and tell it what 
Python script to run. The bash script could look something like this:

\lstinputlisting{Practicals/Code/PythonHPC.sh}

Or, you can do something like this to move all files one-by-one to 
avoid exceeding memory allocation ({\tt *.p} indicates that you used 
{\tt pickle} to dump results):

\begin{lstlisting}
for f in *.p; do
	echo "Processing $f..."
	mv $f $WORK/TestPyHPC/output/
done
\end{lstlisting}

NOTE: PBS  allows  you  to  submit jobs using a Python (instead of 
shell) script as well. Look up the qsub manual ({\tt man qsub}) in the 
HPC terminal, or visit 
\url{https://gist.github.com/nobias/5b2373258e595e5242d5}

Your HPC enabled Python code could look like this:

\lstinputlisting{Practicals/Code/MyHPCScript.py}

% In your Python code you need to set the environment so that it
% knows its working directory and where to output files:

% \begin{lstlisting}
  % home <- os.getenv('HOME')
  % ..
  % save(object, file=`home/whatever/object.RData')
% \end{lstlisting}

\section{Copying scripts from your computer to the HPC server}
 
Secure copy bash script file to {\tt \$HOME} on HPC server following
{\tt \$ scp source host:destination} structure, e.g.:

\begin{lstlisting}
$ scp script.sh user@login.cx1.hpc.ic.ac.uk:/home/user/whatever/script.sh
\end{lstlisting}

\section{Running scripts on the HPC}

Open a secure shell (ssh):

\begin{lstlisting}
$ ssh user@login.cx1.hpc.ic.ac.uk
\end{lstlisting}

Check for available modules:

\begin{lstlisting}
$ module avail
\end{lstlisting}

Your job then needs to be queued using {\tt qsub} (PBS):

\begin{lstlisting}
  $ qsub -j eo script.sh
\end{lstlisting}

where {\tt -j eo} is an option to join both output and error into one 
file. Running the script will produce a job output (anything that is 
printed in the shell terminal (e.g. {\tt echo})), and an error file 
(related to whether the script was successful or not), in the form of 
\{scriptname\}.o\{job id\} and \{scriptname\}.e\{jobid\}.\\

The {\tt qstat} command provides information on the job being submitted 
(which queue (short, medium, long), status, etc.) as well as 
information on all queues available (-q, -Q).
